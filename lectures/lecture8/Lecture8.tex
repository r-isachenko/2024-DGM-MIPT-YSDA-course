\input{../utils/preamble}
\createdgmtitle{8}
%--------------------------------------------------------------------------------
\begin{document}
%--------------------------------------------------------------------------------
\begin{frame}[noframenumbering,plain]
%\thispagestyle{empty}
\titlepage
\end{frame}
%=======
\begin{frame}{Recap of previous lecture}
	\begin{block}{Theorem (Kantorovich-Rubinstein duality)}
		\vspace{-0.2cm}
		\[
		W(\pi || p) = \frac{1}{K} \max_{\| f \|_L \leq K} \left[ \bbE_{\pi(\bx)} f(\bx)  - \bbE_{p(\bx)} f(\bx)\right],
		\]
		where $\| f \|_L \leq K$ are $K-$Lipschitz continuous functions.
	\end{block}
	\begin{block}{WGAN objective}
		\vspace{-0.3cm}
		\[
		\min_{\btheta} {\color{violet}W(\pi || p)} = \min_{\btheta} {\color{violet}\max_{\bphi \in \boldsymbol{\Phi}} \left[ \bbE_{\pi(\bx)} f_{\bphi}(\bx)  - \bbE_{p(\bz)} f_{\bphi}(\bG_{\btheta}(\bz))\right]}.
		\]
		\vspace{-0.3cm}
	\end{block}
	\begin{itemize}
		\item Function~$f$ in WGAN is usually called $\textit{critic}$.
		\item If parameters $\bphi$ lie in a compact set $\boldsymbol{\Phi} \in [-c, c]^d$ then $f(\bx, \bphi)$ will be $K$-Lipschitz continuous function. 
	\end{itemize}
	\begin{multline*}
		K \cdot W(\pi || p) = \max_{\| f \|_L \leq K} \left[ \bbE_{\pi(\bx)} f(\bx)  - \bbE_{p(\bx)} f(\bx)\right] \geq \\  \geq \max_{\bphi \in \boldsymbol{\Phi}} \left[ \bbE_{\pi(\bx)} f_{\bphi}(\bx)  - \bbE_{p(\bx)} f_{\bphi}(\bx)\right]
	\end{multline*}
	\myfootnotewithlink{https://arxiv.org/abs/1701.07875}{Arjovsky M., Chintala S., Bottou L. Wasserstein GAN, 2017}
\end{frame}
%=======
\begin{frame}{Recap of previous lecture}
	\vspace{-0.3cm}
	\begin{block}{f-divergence minimization}
		\vspace{-0.3cm}
		\[
			D_f(\pi || p) = \bbE_{p(\bx)}  f\left( \frac{\pi(\bx)}{p(\bx)} \right) \rightarrow \min_p.
		\]
		Here $f: \bbR_+ \rightarrow \bbR$ is a convex, lower semicontinuous function satisfying $f(1) = 0$.
	\end{block}
	\begin{block}{Variational divergence estimation}
		\vspace{-0.3cm}
		\[
			D_f(\pi || p) \geq \sup_{T \in \cT} \left[\bbE_{\pi}T(\bx) -  \bbE_p f^*(T(\bx)) \right],
		\]
		\vspace{-0.7cm}
	\end{block}
	\begin{block}{Fenchel conjugate}
		\vspace{-0.7cm}
		\[
		f^*(t) = \sup_{u \in \text{dom}_f} \left( ut - f(u) \right), \quad f(u) = \sup_{t \in \text{dom}_{f^*}} \left( ut - f^*(t) \right)
		\]
		\vspace{-0.5cm}
	\end{block}
	\textbf{Note:} To evaluate the lower bound we only need samples from $\pi(\bx)$ and $p(\bx)$. 
	\myfootnotewithlink{https://arxiv.org/abs/1606.00709}{Nowozin S., Cseke B., Tomioka R. f-GAN: Training Generative Neural Samplers using Variational Divergence Minimization, 2016}
\end{frame}
%=======
\begin{frame}{Recap of previous lecture}
	How to evaluate likelihood-free models?
	
	$p(y | \bx)$ -- pretrained image classification model (e.g. ImageNet classifier).
	\begin{block}{What do we want from samples?}
		\begin{itemize}
			\item \textbf{Sharpness}
			\begin{figure}
				\centering
				\includegraphics[width=0.9\linewidth]{figs/sharpness}
			\end{figure}
			$p(y | \bx)$ has low entropy (each image $\bx$ should have distinctly recognizable object).
			\item \textbf{Diversity}
			\begin{figure}
				\centering
				\includegraphics[width=0.9\linewidth]{figs/diversity}
			\end{figure}
			$p(y) = \int p(y | \bx) p(\bx) d \bx$ has high entropy (there should be as many classes generated as possible).
		\end{itemize}
	\end{block}
	\myfootnotewithlink{https://deepgenerativemodels.github.io}{image credit: https://deepgenerativemodels.github.io}
\end{frame}
%=======
\begin{frame}{Recap of previous lecture}
	\vspace{-0.3cm}
	\begin{block}{Frechet Inception Distance (FID)}
		In case of Normal distributions $\pi(\bx) = \cN(\bmu_\pi, \bSigma_\pi)$, $p(\by) = \cN(\bmu_p, \bSigma_p)$:
		\vspace{-0.3cm}
		\begin{multline*}
			\text{FID} (\pi, p) =  W_2^2(\pi, p) = \inf_{\gamma \in \Gamma(\pi, p)} \bbE_{(\bx, \by) \sim \gamma} \| \bx - \by \|^2 \\
			= \| \bmu_{\pi} - \bmu_{p}\|^2 + \text{tr} \left[ \bSigma_{\pi} + \bSigma_p - 2 \left(\bSigma_{\pi}^{1/2} \bSigma_p \bSigma_{\pi}^{1/2} \right)^{1/2} \right]
		\end{multline*}
		\vspace{-0.4cm}
	\end{block}
	\begin{itemize}
		\item Needs a large sample size for evaluation.
		\item Calculation of FID is slow.
		\item High dependence on the pretrained classification model.
		\item Uses the normality assumption!
	\end{itemize}
	\myfootnotewithlink{https://arxiv.org/abs/1706.08500}{Heusel M. et al. GANs Trained by a Two Time-Scale Update Rule Converge to a Local Nash Equilibrium, 2017}
\end{frame}
%=======
\begin{frame}{Recap of previous lecture}
	\vspace{-0.2cm}
	\begin{itemize}
		\item $\cS_{\pi} = \{\bx_i\}_{i=1}^{n} \sim \pi(\bx)$ -- real samples;
		\item $\cS_{p} = \{\bx_i\}_{i=1}^{n} \sim p(\bx | \btheta)$ -- generated samples.
	\end{itemize}
	Define binary function:
	\vspace{-0.2cm}
	\[
		\mathbb{I}(\bx, \cS) = 
		\begin{cases}
			1, \quad \text{if exists } \bx' \in \cS: \| \bx  - \bx'\|_2 \leq \| \bx' - \text{NN}_k(\bx', \cS)\|_2; \\
			0, \quad \text{otherwise.}
		\end{cases}
	\]
	\vspace{-0.3cm}
	\[
		\text{Precision} (\cS_{\pi}, \cS_{p}) = \frac{1}{n} \sum_{\mathbf{\bx} \in \cS_{p}} \mathbb{I}(\bx, \cS_{\pi}); \,\, \text{Recall} (\cS_{\pi}, \cS_{p}) = \frac{1}{n} \sum_{\bx \in \cS_{\pi}} \mathbb{I}(\bx, \cS_{p}).
	\]
	\vspace{-0.6cm}
	\begin{figure}
		\includegraphics[width=0.75\linewidth]{figs/pr_k_nearest}
	\end{figure}
	Embed the samples using the pretrained network (as for FID).
	\myfootnotewithlink{https://arxiv.org/abs/1904.06991}{Kynkäänniemi T. et al. Improved precision and recall metric for assessing generative models, 2019}
\end{frame}
%=======
\begin{frame}{Outline}
	\tableofcontents
\end{frame}
%=======
\section{Langevin dynamic}
%=======
\begin{frame}{Energy-based models}
	\begin{block}{Unnormalized density}
		\vspace{-0.2cm}
		\[
			p(\bx | \btheta) = \frac{\hat{p}(\bx | \btheta)}{Z_{\btheta}}, \quad \text{where } Z_{\btheta} = \int \hat{p}(\bx | \btheta) d \bx
		\]
		\vspace{-0.3cm}
		\begin{itemize}
			\item $\hat{p}(\bx | \btheta)$ is any non-negative function. \\
			\item If we use the reparametrization $\hat{p}(\bx | \btheta) = \exp(-f_{\btheta}(\bx))$, we remove the non-negativite constraint.
		\end{itemize}
	\end{block}
	\vspace{-0.3cm}
	\begin{block}{Unnormalized density}
		The gradient of the normalized log-density equals to the gradient of the unnormalized log-density:
		\[
			\nabla_{\bx} \log p(\bx | \btheta) = \nabla_{\bx} \log \hat{p}(\bx | \btheta) - \nabla_{\bx} \log Z_{\btheta} = \nabla_{\bx} \log \hat{p}(\bx | \btheta)
		\]
	\end{block}
	\vspace{-0.3cm}
	\begin{itemize}
		\item Let assume that we already have the density (normalized or unnormalized) $p(\bx | \btheta)$.
		\item How to sample from the model?
	\end{itemize}
\end{frame}
%=======
\begin{frame}{Langevin dynamic}
	\vspace{-0.4cm}
	\begin{block}{Theorem (informal)}
		Let $\bx_0$ be a random vector. Under some mild regularity conditions samples from the following dynamics will come from $p(\bx | \btheta)$ (for small enough $\eta$ and large enough $l$)
		\vspace{-0.3cm}
		\[
			\bx_{l + 1} = \bx_l + \frac{\eta}{2} \cdot \nabla_{\bx_l} \log p(\bx_l | \btheta) + \sqrt{\eta} \cdot \bepsilon_l, \quad \bepsilon_l \sim \cN(0, \bI).
		\]
		\vspace{-0.5cm}
	\end{block}
	\begin{minipage}{0.55\linewidth}
		\begin{itemize}
			\item What do we get if $\bepsilon_l = \boldsymbol{0}$?
			\item The density $p(\bx | \btheta)$ is a \textbf{stationary} distribution for the Markov chain. 
			\item We take the gradient w.r.t. to $\bx$ (not $\btheta$).
			\item $\nabla_{\bx} \log p(\bx | \btheta)$ defines the vector field.
		\end{itemize}
	\end{minipage}%
	\begin{minipage}{0.45\linewidth}
		\begin{figure}
			\centering
			\includegraphics[width=0.9\linewidth]{figs/langevin_dynamic}
		\end{figure}
	\end{minipage}
	\myfootnotewithlink{https://yang-song.github.io/blog/2021/score/}{Song Y. Generative Modeling by Estimating Gradients of the Data Distribution, blog post, 2021} 
	\end{frame}
%=======
\section{Score matching}
%=======
\begin{frame}{Score matching}
	\begin{block}{Score function}
		\vspace{-0.3cm}
		 \[
			 \bs_{\btheta}(\bx) = \nabla_{\bx}\log p(\bx| \btheta)
		 \]
		\vspace{-0.5cm} 
	\end{block}
	\begin{block}{Langevin dynamic}
		If we find the score function $\bs_{\btheta}(\bx) = \nabla_{\bx}\log p(\bx| \btheta)$ we will be able to sample from the model using Langevin dynamic. 
		\[
			\bx_{l + 1} = \bx_l + \frac{\eta}{2} \cdot {\color{violet}\nabla_{\bx_l} \log p(\bx_l | \btheta)} + \sqrt{\eta} \cdot \bepsilon_l = \bx_l + \frac{\eta}{2} \cdot  {\color{violet}\bs_{\btheta}(\bx_l)} + \sqrt{\eta} \cdot \bepsilon_l.
		\] 
		\vspace{-0.5cm} 
	\end{block}
	\begin{block}{Fisher divergence}
		\vspace{-0.5cm}
		\begin{multline*}
			D_F(\pi, p) = \frac{1}{2}\bbE_{\pi}\bigl\| \nabla_{\bx}\log p(\bx| \btheta) - \nabla_\bx \log \pi(\bx) \bigr\|^2_2 = \\ 
			= \frac{1}{2}\bbE_{\pi}\bigl\| \bs_{\btheta}(\bx) - \nabla_\bx \log \pi(\bx) \bigr\|^2_2 \rightarrow \min_{\btheta}
		\end{multline*}
		\vspace{-0.3cm}
	\end{block}
	\myfootnotewithlink{https://yang-song.github.io/blog/2021/score/}{Song Y. Generative Modeling by Estimating Gradients of the Data Distribution, blog post, 2021}
\end{frame}
%=======
\begin{frame}{Score matching}
	\begin{block}{Fisher divergence}
		\vspace{-0.3cm}
		\[
			D_F(\pi, p) = \frac{1}{2}\bbE_{\pi}\bigl\| \bs_{\btheta}(\bx) - \nabla_\bx \log \pi(\bx) \bigr\|^2_2 \rightarrow \min_{\btheta}
		\]
		\vspace{-0.5cm}
	\end{block}
	\begin{figure}
		\centering
		\includegraphics[width=\linewidth]{figs/smld}
	\end{figure} 
	\textbf{Problem:} We do not know $\nabla_\bx \log \pi(\bx)$.
	\myfootnotewithlink{https://yang-song.github.io/blog/2021/score/}{Song Y. Generative Modeling by Estimating Gradients of the Data Distribution, blog post, 2021}
\end{frame}
%=======
\subsection{Denoising score matching}
%=======
\begin{frame}{Denoising score matching}
	Let perturb original data $\bx \sim \pi(\bx)$ by random normal noise 
	\[
		\bx_{\sigma} = \bx + \sigma \cdot \bepsilon, \quad \bepsilon \sim \cN(0, \bI), \quad q(\bx_{\sigma} | \bx) = \cN(\bx, \sigma^2 \cdot \bI)
	\]
	\vspace{-0.4cm}
	\[
		q(\bx_{\sigma}) = \int q(\bx_{\sigma} | \bx) \pi(\bx) d\bx.
	\]
	\vspace{-0.5cm} 
	\begin{block}{Assumption}
		The solution of 
		\[
			\frac{1}{2} \bbE_{q(\bx_{\sigma})}\bigl\| \bs_{\btheta, \sigma}(\bx_{\sigma}) - \nabla_{\bx_{\sigma}} \log q(\bx_{\sigma}) \bigr\|^2_2 \rightarrow \min_{\btheta}
		\]
		\vspace{-0.3cm} \\
		satisfies $\bs_{\btheta, \sigma}(\bx_{\sigma}) \approx \bs_{\btheta, 0}(\bx_0) = \bs_{\btheta}(\bx)$ if $\sigma$ is small enough.
	\end{block}
	\begin{itemize}
		\item The score function of the noised data is almost the same as the score function of the original data.
		\item Score function $\bs_{\btheta, \sigma}(\bx_{\sigma})$ parametrized by $\sigma$. 
		\item \textbf{Note:} We don't know $q(\bx_{\sigma})$, just like $\pi(\bx)$.
	\end{itemize}
	\myfootnotewithlink{http://www.iro.umontreal.ca/~vincentp/Publications/smdae_techreport.pdf}{Vincent P. A Connection Between Score Matching and Denoising Autoencoders, 2010}
\end{frame}
%=======
\begin{frame}{Denoising score matching}
	\begin{block}{Theorem}
	\vspace{-0.5cm}
	\begin{multline*}
		\bbE_{q(\bx_{\sigma})}\bigl\| \bs_{\btheta, \sigma}(\bx_{\sigma}) - \nabla_{\bx_{\sigma}} \log q(\bx_{\sigma}) \bigr\|^2_2 = \\
		= \bbE_{\pi(\bx)} \bbE_{q(\bx_{\sigma} | \bx)}\bigl\| \bs_{\btheta, \sigma}(\bx_{\sigma}) - \nabla_{\bx_{\sigma}} \log q(\bx_{\sigma} | \bx) \bigr\|^2_2 + \text{const}(\btheta)
	\end{multline*}
	\vspace{-0.5cm}
	\end{block}
	\begin{block}{Gradient of the noise kernel}
		\vspace{-0.3cm}
		\[
			\bx_{\sigma} = \bx + \sigma \cdot \bepsilon, \quad q(\bx_{\sigma} | \bx) = \cN(\bx, \sigma^2 \cdot \bI)
		\]
		\vspace{-0.3cm}
		\[
			\nabla_{\bx_{\sigma}} \log q(\bx_{\sigma} | \bx) = - \frac{\bx_{\sigma} - \bx}{\sigma^2}  = - \frac{\bepsilon}{\sigma}
		\]
		\vspace{-0.5cm}
	\end{block}
	\begin{itemize}
		\item The RHS does not need to compute $\nabla_{\bx_{\sigma}} \log q(\bx_{\sigma})$ and even $\nabla_{\bx_{\sigma}} \log \pi(\bx_{\sigma})$.
		\item $\bs_{\btheta, \sigma}(\bx_{\sigma})$ tries to \textbf{denoise} the noised samples $\bx_{\sigma}$. 
	\end{itemize}
	\myfootnotewithlink{http://www.iro.umontreal.ca/~vincentp/Publications/smdae_techreport.pdf}{Vincent P. A Connection Between Score Matching and Denoising Autoencoders, 2010}
\end{frame}
%=======
\begin{frame}{Denoising score matching}
	\begin{block}{Theorem}
	\vspace{-0.5cm}
	\begin{multline*}
		\bbE_{q(\bx_{\sigma})}\underbrace{\bigl\| \bs_{\btheta, \sigma}(\bx_{\sigma}) - \nabla_{\bx_{\sigma}} \log q(\bx_{\sigma}) \bigr\|^2_2}_{h(\bx_{\sigma})} = \\
		= \bbE_{\pi(\bx)} \bbE_{q(\bx_{\sigma} | \bx)}\bigl\| \bs_{\btheta, \sigma}(\bx_{\sigma}) - \nabla_{\bx_{\sigma}} \log q(\bx_{\sigma} | \bx) \bigr\|^2_2 + \text{const}(\btheta)
	\end{multline*}
	\vspace{-0.5cm}
	\end{block}
	\begin{block}{Proof}
		\vspace{-0.7cm}
		\begin{multline*}
			\bbE_{q(\bx_{\sigma})} h(\bx_{\sigma}) = \int {\color{violet}q(\bx_{\sigma})} h(\bx_{\sigma}) d\bx_{\sigma} = \\
			= \int \left({\color{violet}\int q(\bx_{\sigma} | \bx) \pi(\bx) d\bx}\right) h(\bx_{\sigma}) d\bx_{\sigma} =  \bbE_{\pi(\bx)} \bbE_{q(\bx_{\sigma} | \bx)}  h(\bx_{\sigma})
		\end{multline*}
		\vspace{-0.7cm}
		{\small
		\begin{multline*}
			\bbE_{q(\bx_{\sigma})}\bigl\| \bs_{\btheta, \sigma}(\bx_{\sigma}) - \nabla_{\bx_{\sigma}} \log q(\bx_{\sigma}) \bigr\|^2_2 = \\ 
			= \bbE_{q(\bx_{\sigma})} \Bigl[\| \bs_{\btheta, \sigma}(\bx_{\sigma}) \|^2 + \underbrace{\| \nabla_{\bx_{\sigma}} \log q(\bx_{\sigma}) \|^2_2}_{\text{const}(\btheta)} - 2 {\color{teal}\bs_{\btheta, \sigma}^T(\bx_{\sigma}) \nabla_{\bx_{\sigma}} \log q(\bx_{\sigma})} \Bigr]
		\end{multline*}
		}
	\end{block}
	\myfootnotewithlink{http://www.iro.umontreal.ca/~vincentp/Publications/smdae_techreport.pdf}{Vincent P. A Connection Between Score Matching and Denoising Autoencoders, 2010}
\end{frame}
%=======
\begin{frame}{Denoising score matching}
	\begin{block}{Theorem}
	\vspace{-0.5cm}
	\begin{multline*}
		\bbE_{q(\bx_{\sigma})}\bigl\| \bs_{\btheta, \sigma}(\bx_{\sigma}) - \nabla_{\bx_{\sigma}} \log q(\bx_{\sigma}) \bigr\|^2_2 = \\
		= \bbE_{\pi(\bx)} \bbE_{q(\bx_{\sigma} | \bx)}\bigl\| \bs_{\btheta, \sigma}(\bx_{\sigma}) - \nabla_{\bx_{\sigma}} \log q(\bx_{\sigma} | \bx) \bigr\|^2_2 + \text{const}(\btheta)
	\end{multline*}
	\vspace{-0.5cm}
	\end{block}
	\begin{block}{Proof (continued)}
		\vspace{-0.7cm}
		{\small
		\begin{multline*}
			\bbE_{q(\bx_{\sigma})} \left[{\color{teal}\bs_{\btheta, \sigma}^T(\bx_{\sigma}) \nabla_{\bx_{\sigma}} \log q(\bx_{\sigma})} \right] = \int q(\bx_{\sigma}) \left[\bs_{\btheta, \sigma}^T(\bx_{\sigma}) \frac{\nabla_{\bx_{\sigma}} {\color{violet}q(\bx_{\sigma})}}{q(\bx_{\sigma})} \right] d\bx_{\sigma} = \\
			= \int \left[\bs_{\btheta, \sigma}^T(\bx_{\sigma}) \nabla_{\bx_{\sigma}}\left({\color{violet}\int q(\bx_{\sigma} | \bx) \pi(\bx) d\bx}\right) \right] d\bx_{\sigma} = \\
			=  \int \int \pi(\bx) \left[\bs_{\btheta, \sigma}^T(\bx_{\sigma}) {\color{olive}\nabla_{\bx_{\sigma}}q(\bx_{\sigma} | \bx)} \right] d\bx_{\sigma} d\bx = \\
			= \int \int \pi(\bx) {\color{olive} q(\bx_{\sigma} | \bx)} \left[\bs_{\btheta, \sigma}^T(\bx_{\sigma}) {\color{olive}\nabla_{\bx_{\sigma}} \log q(\bx_{\sigma} | \bx)} \right] d\bx_{\sigma} d\bx = \\
			= \bbE_{\pi(\bx)} \bbE_{q(\bx_{\sigma} | \bx)} \left[{\color{teal}\bs_{\btheta, \sigma}^T(\bx_{\sigma}) \nabla_{\bx_{\sigma}} \log q(\bx_{\sigma} | \bx)} \right]
		\end{multline*}
		}
	\end{block}
	\myfootnotewithlink{http://www.iro.umontreal.ca/~vincentp/Publications/smdae_techreport.pdf}{Vincent P. A Connection Between Score Matching and Denoising Autoencoders, 2010}
\end{frame}
%=======
\begin{frame}{Denoising score matching}
	\vspace{-0.3cm}
	\begin{block}{Theorem}
	\vspace{-0.7cm}
	\begin{multline*}
		\bbE_{q(\bx_{\sigma})}\underbrace{\bigl\| \bs_{\btheta, \sigma}(\bx_{\sigma}) - \nabla_{\bx_{\sigma}} \log q(\bx_{\sigma}) \bigr\|^2_2}_{h(\bx_{\sigma})} = \\
		= \bbE_{\pi(\bx)} \bbE_{q(\bx_{\sigma} | \bx)}\bigl\| \bs_{\btheta, \sigma}(\bx_{\sigma}) - \nabla_{\bx_{\sigma}} \log q(\bx_{\sigma} | \bx) \bigr\|^2_2 + \text{const}(\btheta)
	\end{multline*}
	\vspace{-0.9cm}
	\end{block}
	\begin{block}{Proof (continued)}
		\vspace{-0.3cm}
		\[
			\bbE_{q(\bx_{\sigma})} h(\bx_{\sigma})=  \bbE_{\pi(\bx)} \bbE_{q(\bx_{\sigma} | \bx)}  h(\bx_{\sigma}) d\bx_{\sigma}
		\]
		\[
			\bbE_{q(\bx_{\sigma})} \left[\bs_{\btheta, \sigma}^T(\bx_{\sigma}) \nabla_{\bx_{\sigma}} \log q(\bx_{\sigma}) \right] = \bbE_{\pi(\bx)} \bbE_{q(\bx_{\sigma} | \bx)} \left[\bs_{\btheta, \sigma}^T(\bx_{\sigma}) \nabla_{\bx_{\sigma}} \log q(\bx_{\sigma} | \bx) \right]
		\]
		\vspace{-0.5cm}
		{\small
		\begin{multline*}
			\bbE_{q(\bx_{\sigma})}\bigl\| \bs_{\btheta, \sigma}(\bx_{\sigma}) - \nabla_{\bx_{\sigma}} \log q(\bx_{\sigma}) \bigr\|^2_2 = \\ 
			= {\color{olive}\bbE_{\pi(\bx)} \bbE_{q(\bx_{\sigma} | \bx)}}\Bigl[\| \bs_{\btheta, \sigma}(\bx_{\sigma}) \|^2 - 2 \bs_{\btheta, \sigma}^T(\bx_{\sigma}) {\color{teal}\nabla_{\bx_{\sigma}} \log q(\bx_{\sigma} | \bx)} \Bigr] + \text{const}(\btheta) \\
			= {\color{olive}\bbE_{\pi(\bx)} \bbE_{q(\bx_{\sigma} | \bx)}} \bigl\|\bs_{\btheta, \sigma}(\bx_{\sigma}) - {\color{teal}\nabla_{\bx_{\sigma}} \log q(\bx_{\sigma} | \bx)} \bigr\|_2^2 + \text{const}(\btheta)
		\end{multline*}
		}
		\vspace{-0.8cm}
	\end{block}
	\myfootnotewithlink{http://www.iro.umontreal.ca/~vincentp/Publications/smdae_techreport.pdf}{Vincent P. A Connection Between Score Matching and Denoising Autoencoders, 2010}
\end{frame}
%=======
\begin{frame}{Denoising score matching}
	Initial objective:
	\vspace{-0.2cm}
	\[
		\bbE_{\pi}\bigl\| \bs_{\btheta}(\bx) - \nabla_\bx \log \pi(\bx) \bigr\|^2_2 \rightarrow \min_{\btheta}
	\]
	\vspace{-0.5cm} \\
	Noised objective:
	\vspace{-0.2cm}
	\[
		\bbE_{\pi}\bigl\| \bs_{\btheta, \sigma}(\bx_\sigma) - \nabla_\bx \log q(\bx_{\sigma}) \bigr\|^2_2 \rightarrow \min_{\btheta}
	\]
	\vspace{-0.5cm} \\
	This is equivalent to denoising task
	\vspace{-0.2cm}
	\[
		\bbE_{\pi(\bx)} \bbE_{q(\bx_{\sigma} | \bx)}\bigl\| \bs_{\btheta, \sigma}(\bx_{\sigma}) - \nabla_{\bx_{\sigma}} \log q(\bx_{\sigma} | \bx) \bigr\|^2_2 \rightarrow \min_{\btheta}
	\]
	\vspace{-0.3cm}
	\[
		\bbE_{\pi(\bx)} \bbE_{\cN(0, \bI)}\left\| \bs_{\btheta, \sigma}(\bx + \sigma \cdot \bepsilon) + \frac{\bepsilon}{\sigma} \right\|^2_2 \rightarrow \min_{\btheta}
	\]
	\vspace{-0.5cm}
	\begin{block}{Langevin dynamic}
		\vspace{-0.3cm}
		\[
			\bx_{l + 1} = \bx_l + \frac{\eta}{2} \cdot \bs_{\btheta, \sigma}(\bx_l) + \sqrt{\eta} \cdot \bepsilon_l, \quad \bepsilon_l \sim \cN(0, \bI).
		\]
		\vspace{-0.7cm}
	\end{block}
	\myfootnotewithlink{https://yang-song.github.io/blog/2021/score/}{Song Y. Generative Modeling by Estimating Gradients of the Data Distribution, blog post, 2021}
\end{frame}
%=======
\subsection{Noise Conditioned Score Network (NCSN)}
%=======
\begin{frame}{Denoising score matching}
	\vspace{-0.5cm}
	\[
		\bbE_{\pi(\bx)} \bbE_{\cN(0, \bI)}\left\| \bs_{\btheta, \sigma}(\bx + \sigma \cdot \bepsilon) + \frac{\bepsilon}{\sigma} \right\|^2_2 \rightarrow \min_{\btheta}
	\]
	\begin{minipage}{0.5\linewidth}
		\begin{itemize}
			\item If $\sigma$ is \textbf{small}, the score function is not accurate and Langevin dynamics will probably fail to jump between modes.
			\item If $\sigma$ is \textbf{large}, it is good for low-density regions and  multimodal distributions, but we will learn too corrupted distribution.
		\end{itemize}
	\end{minipage}%
	\begin{minipage}{0.5\linewidth}
		\begin{figure}
			\includegraphics[width=\linewidth]{figs/pitfalls}
		\end{figure}
		\vspace{-0.3cm}
		\begin{figure}
			\includegraphics[width=\linewidth]{figs/single_noise}
		\end{figure}
	\end{minipage}
	\myfootnotewithlink{https://yang-song.github.io/blog/2021/score/}{Song Y. Generative Modeling by Estimating Gradients of the Data Distribution, blog post, 2021}
\end{frame}
%=======
\begin{frame}{Noise Conditioned Score Network (NCSN)}
	\begin{itemize}
		\item Define the sequence of the noise levels: $\sigma_1 < \sigma_2 < \dots < \sigma_T$.
		\item Perturb the original data with the different noise levels to obtain $\bx_t = \bx + \sigma_t \cdot \bepsilon$, $\bx_t \sim q(\bx_t)$. 
		\item Choose $\sigma_1, \sigma_T$ such that:
		\[
			q(\bx_1) \approx \pi(\bx), \quad q(\bx_T) \approx \cN(0, \sigma_T^2 \cdot \bI).
		\]
	\end{itemize}
	\begin{figure}
		\includegraphics[width=0.6\linewidth]{figs/multi_scale}
	\end{figure}
	\begin{figure}
		\includegraphics[width=\linewidth]{figs/duoduo}
	\end{figure}
	\myfootnotewithlink{https://arxiv.org/abs/1907.05600}{Song Y. et al. Generative Modeling by Estimating Gradients of the Data Distribution, 2019}
\end{frame}
%=======
\begin{frame}{Noise Conditioned Score Network (NCSN)}
	Train the denoising score function $\bs_{\btheta, \sigma_t}(\bx_t)$ for each noise level $\sigma_t$ using unified weighted objective:
	\vspace{-0.2cm}
	\[
		\sum_{t=1}^T {\color{violet}\sigma_t^2} \cdot \bbE_{\pi(\bx)} \bbE_{q(\bx_t | \bx)}\bigl\| \bs_{\btheta, \sigma_t}(\bx_t) - \nabla_{\bx_t} \log q(\bx_t | \bx) \bigr\|^2_2 \rightarrow \min_{\btheta}
	\]
	Here $\nabla_{\bx_t} \log q(\bx_t | \bx) = - \frac{\bx_t - \bx}{\sigma_t^2} = - \frac{\bepsilon}{\sigma_t}$.
	\begin{block}{Training}
		\begin{enumerate}
			\item Get the sample $\bx_0 \sim \pi(\bx)$.
			\item Sample noise level $t \sim U\{1, T\}$ and the noise $\bepsilon \sim \cN(0, \bI)$.
			\item Get noisy image $\bx_t = \bx_0 + \sigma_t \cdot \bepsilon$.
			\item Compute loss $ \cL = \sigma_t^2 \cdot \| \bs_{\btheta, \sigma_t}(\bx_t) + \frac{\bepsilon}{\sigma_t} \|^2 $.
		\end{enumerate}
	\end{block}
	How to sample from this model?
	\myfootnotewithlink{https://arxiv.org/abs/1907.05600}{Song Y. et al. Generative Modeling by Estimating Gradients of the Data Distribution, 2019}
\end{frame}
%=======
\begin{frame}{Noise Conditioned Score Network (NCSN)}
	\begin{block}{Sampling (annealed Langevin dynamics)}
		\begin{itemize}
			\item Sample $\bx_0 \sim \cN(0, \sigma_T^2 \cdot \bI) \approx q(\bx_T)$.
			\item Apply $L$ steps of Langevin dynamic
			\vspace{-0.2cm}
			\[
				\bx_l = \bx_{l-1} + \frac{\eta_t}{2} \cdot \bs_{\btheta, \sigma_t}(\bx_{l - 1}) + \sqrt{\eta_t} \cdot \bepsilon_l.
			\] 
			\vspace{-0.5cm}
			\item Update $\bx_0 := \bx_L$ and choose the next $\sigma_t$.
		\end{itemize}
	\end{block}
	\begin{figure}
		\includegraphics[width=0.9\linewidth]{figs/ald}
	\end{figure}
	\myfootnotewithlink{https://arxiv.org/abs/2006.09011}{Song Y. et al. Improved Techniques for Training Score-Based Generative Models, 2020}
\end{frame}
%=======
\section{Forward gaussian diffusion process}
%=======
\begin{frame}{Forward gaussian diffusion process}
	Let $\bx_0 = \bx \sim \pi(\bx)$, $\beta_t \ll 1$. Define the Markov chain
	\[
		\bx_t = \sqrt{1 - \beta_t} \cdot \bx_{t - 1} + \sqrt{\beta_t} \cdot \bepsilon_t, \quad \text{where }\bepsilon_t \sim \cN(0, \bI);
	\]
	\[
		q(\bx_t | \bx_{t-1}) = \cN(\sqrt{1 - \beta_t} \cdot \bx_{t-1}, \beta_t \cdot \bI).
	\]
	\vspace{-0.5cm}
	\begin{block}{Langevin dynamics}
		\vspace{-0.3cm}
		\[
			\bx_{l + 1} = \bx_l + \frac{\color{violet} \eta}{2} \cdot {\color{teal}\nabla_{\bx_l} \log p(\bx_l | \btheta)} + \sqrt{\color{violet} \eta} \cdot \bepsilon_l, \quad \bepsilon_l \sim \cN(0, \bI).
		\]
		\vspace{-0.5cm}
	\end{block}
	\vspace{-0.7cm}
	\begin{multline*}
		\bx_t = \sqrt{1 - \beta_t} \cdot \bx_{t - 1} + \sqrt{\beta_t} \cdot \bepsilon_t \approx \left(1 - \frac{\beta_t}{2}\right) \cdot \bx_{t - 1} + \sqrt{\beta_t} \cdot \bepsilon_t = \\ =  \bx_{t - 1} + \frac{ \color{violet}  \beta_t}{2} \cdot  {\color{teal} (-\bx_{t - 1})} + \sqrt{ \color{violet}  \beta_t} \cdot \bepsilon_t
	\end{multline*}
	\vspace{-1.0cm}
	\begin{itemize}
		\item ${\color{violet} \beta_t = \eta}$
		\item ${\color{teal} \nabla_{\bx_{t-1}}\log p(\bx_{t-1} | \btheta) = - \bx_{t-1}} = \nabla_{\bx_{t-1}} \log \cN(0, \bI)$
	\end{itemize}
	\myfootnotewithlink{http://proceedings.mlr.press/v37/sohl-dickstein15.pdf}{Sohl-Dickstein J. Deep Unsupervised Learning using Nonequilibrium Thermodynamics, 2015}
 \end{frame}
%=======
\begin{frame}{Forward gaussian diffusion process}
	\[
		\bx_t = \sqrt{1 - \beta_t} \cdot \bx_{t - 1} + \sqrt{\beta_t} \cdot \bepsilon_t, \quad \text{where }\bepsilon_t \sim \cN(0, \bI);
	\]
	\[
		q(\bx_t | \bx_{t-1}) = \cN(\sqrt{1 - \beta_t} \cdot \bx_{t-1}, \beta_t \cdot \bI).
	\]
	\vspace{-0.5cm}
	\begin{block}{Statement 1}
		Let denote $\alpha_t = 1 - \beta_t$ and $\bar{\alpha}_t = \prod_{s=1}^t \alpha_s = \prod_{s=1}^t (1 - \beta_s)$. Then
		\[
			q(\bx_t | \bx_0) = \cN(\sqrt{\bar{\alpha}_t} \cdot \bx_0, (1 - \bar{\alpha}_t) \cdot \bI)
		\]
		We are able to sample from any timestamp using only $\bx_0$!
		\vspace{-0.2cm}
		{\small
		\begin{multline*}
			\bx_t = \sqrt{\alpha_t} \cdot {\color{teal}\bx_{t-1}} + \sqrt{1 - \alpha_t} \cdot \boldsymbol{\epsilon}_t = \\
			= \sqrt{\alpha_t} ({\color{teal} \sqrt{\alpha_{t-1}} \cdot \bx_{t-2} + \sqrt{1 - \alpha_{t-1}} \cdot  \boldsymbol{\epsilon}_{t-1}}) + \sqrt{1 - \alpha_t} \cdot \boldsymbol{\epsilon}_t = \\
			= \sqrt{\alpha_t \alpha_{t-1}} \cdot \bx_{t-2} + ( {\color{violet}\sqrt{\alpha_t (1 - \alpha_{t-1})} \cdot  \boldsymbol{\epsilon}_{t-1} + \sqrt{1 - \alpha_t} \cdot \boldsymbol{\epsilon}_t}) = \\
			= \sqrt{\alpha_t \alpha_{t-1}} \cdot \bx_{t-2} + {\color{violet}\sqrt{1 - \alpha_{t-1} \alpha_t} \cdot \boldsymbol{\epsilon}'_t} = \\
			 = \dots = \sqrt{\bar{\alpha}_t} \cdot \bx_{0} + \sqrt{1 - \bar{\alpha}_t} \cdot \bepsilon, \quad \text{where } \bepsilon \sim \cN(0, \bI).
		\end{multline*}
		}
	\end{block}
	\myfootnotewithlink{http://proceedings.mlr.press/v37/sohl-dickstein15.pdf}{Sohl-Dickstein J. Deep Unsupervised Learning using Nonequilibrium Thermodynamics, 2015}
 \end{frame}
%=======
\begin{frame}{Forward gaussian diffusion process}
	\vspace{-0.7cm}
	{\small
	\[
		q(\bx_t | \bx_{t-1}) = \cN\left(\sqrt{1 - \beta_t} \bx_{t-1}, \beta_t \bI\right); \quad q(\bx_t | \bx_0) = \cN\left(\sqrt{\bar{\alpha}_t} \bx_0, (1 - \bar{\alpha}_t) \bI\right).
	\]
	}
	\vspace{-0.8cm}
	\begin{figure}
		\includegraphics[width=0.8\linewidth]{figs/conditional_diffusion}
	\end{figure}
	\vspace{-0.5cm}
	\begin{block}{Statement 2}
		Applying the Markov chain to samples from any $\pi(\bx)$ we will get $\bx_{\infty} \sim p_{\infty}(\bx) = \cN(0, \bI)$. Here $p_{\infty}(\bx)$ is a \textbf{stationary} and \textbf{limiting} distribution:
		\[
			p_{\infty}(\bx) = \int q(\bx | \bx') p_{\infty}(\bx') d \bx'. 
		\]
		\[
			p_{\infty}(\bx) = \int q(\bx_{\infty} | \bx_0) \pi(\bx_0) d\bx_0 \approx \cN(0, \bI) \int \pi(\bx_0) d\bx_0 = \cN(0, \bI)
		\]
		\vspace{-0.8cm}
	\end{block}
	\myfootnote{\href{https://arxiv.org/abs/2403.18103}{Chan S. Tutorial on Diffusion Models for Imaging and Vision, 2024} \\ \href{http://proceedings.mlr.press/v37/sohl-dickstein15.pdf}{Sohl-Dickstein J. Deep Unsupervised Learning using Nonequilibrium Thermodynamics, 2015}}
 \end{frame}
%=======
\begin{frame}{Forward gaussian diffusion process}
	\textbf{Diffusion} refers to the flow of particles from high-density regions towards low-density regions.
	\vspace{-0.2cm}
	\begin{figure}
		\includegraphics[width=0.5\linewidth]{figs/diffusion_over_time}
	\end{figure}
	\vspace{-0.6cm}
	\begin{enumerate}
		\item $\bx_0 = \bx \sim \pi(\bx)$;
		\item $\bx_t = \sqrt{1 - \beta_t} \cdot \bx_{t - 1} + \sqrt{\beta_t} \cdot \bepsilon$, where $\bepsilon \sim \cN(0, \bI)$, $t \geq 1$;
		\item $\bx_T \sim p_{\infty}(\bx) = \cN(0, \bI)$, where $T \gg 1$.
	\end{enumerate}
	If we are able to invert this process, we will get the way to sample $\bx \sim \pi(\bx)$ using noise samples $p_{\infty}(\bx) = \cN(0, \mathbf{I})$. \\ 
	Now our goal is to revert this process.
	\myfootnotewithlink{https://ayandas.me/blog-tut/2021/12/04/diffusion-prob-models.html}{Das A. An introduction to Diffusion Probabilistic Models, blog post, 2021}
\end{frame}
%=======
\begin{frame}{Summary}
	\begin{itemize}
		\item Langevin dynamics allows to sample from the generative model using the gradient of the log-likelihood.	
		\vfill
		\item Score matching proposes to minimize the Fisher divergence to get the score function.
		\vfill 
		\item Denoising score matching minimizes the Fisher divergence on noisy samples. It allows to estimate the Fisher divergence using samples.
		\vfill
		\item Noise conditioned score network uses multiple noise levels and annealed Langevin dynamics to fit the score function and sample from the model.	
		\vfill
		\item Gaussian diffusion process is a Markov chain that injects special form of Gaussian noise to the samples.
	\end{itemize}
\end{frame}
%=======
\end{document} 