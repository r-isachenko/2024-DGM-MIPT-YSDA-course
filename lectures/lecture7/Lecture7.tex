\input{../utils/preamble}
\createdgmtitle{7}
\usepackage{tikz}

\usetikzlibrary{arrows,shapes,positioning,shadows,trees}
%--------------------------------------------------------------------------------
\begin{document}
%--------------------------------------------------------------------------------
\begin{frame}[noframenumbering,plain]
%\thispagestyle{empty}
\titlepage
\end{frame}
%=======
\begin{frame}{Recap of previous lecture}
	\begin{block}{Likelihood-free learning}
		\begin{itemize}
			\item Likelihood is not a perfect quality measure for generative model.
			\item Likelihood could be intractable.
		\end{itemize}
	\end{block}
	Imagine we have two sets of samples 
	\begin{itemize}
		\item $\cS_1 = \{\bx_i\}_{i=1}^{n_1} \sim \pi(\bx)$ -- real samples;
		\item $\cS_2 = \{\bx_i\}_{i=1}^{n_2} \sim p(\bx | \btheta)$ -- generated (or fake) samples.
	\end{itemize}
	Let define discriminative model (classifier):
	\[
		p(y = 1 | \bx) = P\bigl(\{\bx \sim \pi(\bx)\}\bigr); \quad p(y = 0 | \bx) = P\bigl(\{\bx \sim p(\bx | \btheta)\}\bigr)
	\]
	\vspace{-0.5cm}
	\begin{block}{Assumption}
		Generative distribution $p(\bx | \btheta)$ equals to the true distribution $\pi(\bx)$ if we can not distinguish them using discriminative model $p(y | \bx)$. \\
		It means that $p(y = 1 | \bx) = 0.5$ for each sample $\bx$.
	\end{block}
\end{frame}
%=======
\begin{frame}{Recap of previous lecture}
	\begin{itemize}
		\item \textbf{Generator:} generative model $\bx = \bG(\bz)$, which makes generated sample more realistic.
		\item \textbf{Discriminator:} a classifier $D(\bx) \in [0, 1]$, which distinguishes real samples from generated samples.
	\end{itemize}
	\vspace{-0.1cm}
	\begin{block}{GAN optimality theorem}
		The minimax game 
		\vspace{-0.3cm}
		\[
		\min_{G} \max_D \Bigl[ \underbrace{\bbE_{\pi(\bx)} \log D(\bx) + \bbE_{p(\bz)} \log (1 - D(\bG(\bz)))}_{V(G, D)} \Bigr]
		\]
		\vspace{-0.5cm} \\
		has the global optimum $\pi(\bx) = p(\bx | \btheta)$, in this case $D^*(\bx) = 0.5$.
	\end{block}
	\[
		\min_{G} V(G, D^*) = \min_{G} \left[ 2 JSD(\pi || p) - \log 4 \right] = -\log 4, \quad \pi(\bx) = p(\bx | \btheta).
	\]
	\vspace{-0.5cm} \\
	If the generator could be \textbf{any} function and the discriminator is \textbf{optimal} at every step, then the generator is \textbf{guaranteed to converge} to the data distribution.
	 \myfootnotewithlink{https://arxiv.org/abs/1406.2661}{Goodfellow I. J. et al. Generative Adversarial Networks, 2014}
\end{frame}
%=======
\begin{frame}{Recap of previous lecture}
	\begin{itemize}
		\item Generator updates are made in parameter space, discriminator is not optimal at every step.
		\item Generator and discriminator loss keeps oscillating during GAN training.
	\end{itemize}
	\begin{block}{Objective}
		\vspace{-0.5cm}
		\[
		\min_{\btheta} \max_{\bphi} \left[ \bbE_{\pi(\bx)} \log D_{\bphi}(\bx) + \bbE_{p(\bz)} \log (1 - D_{\bphi}(\bG_{\btheta}(\bz))) \right]
		\]
		\vspace{-0.5cm}
	\end{block}
	\begin{figure}
		\centering
		\includegraphics[width=1.0\linewidth]{figs/gan_1}
	\end{figure}
	\myfootnotewithlink{https://arxiv.org/abs/1406.2661}{Goodfellow I. J. et al. Generative Adversarial Networks, 2014}
\end{frame}
%=======
\begin{frame}{Recap of previous lecture}
	\begin{block}{Main problems of standard GAN}
		\begin{itemize}
			\item Vanishing gradients (solution: non-saturating GAN);
			\item Mode collapse (caused by Jensen-Shannon divergence).
		\end{itemize}
	\end{block}
	\begin{block}{Standard GAN}
		\vspace{-0.2cm}
		\[
		\min_{\btheta} \max_{\bphi} \left[ \bbE_{\pi(\bx)} \log D_{\bphi}(\bx) + \bbE_{p(\bz)} \log (1 - D_{\bphi}(\bG_{\btheta}(\bz))) \right]
		\]
		\vspace{-0.4cm}
	\end{block}
	\vspace{-0.1cm}
	\begin{block}{Informal theoretical results}
		The real images distribution $\pi(\bx)$ and the generated images distribution $p(\bx | \btheta)$ are low-dimensional and have disjoint supports. In this case
		\vspace{-0.3cm}
		\[
		KL(\pi || p) = KL(p || \pi) = \infty, \quad JSD(\pi || p) = \log 2.
		\]
	\end{block}
	\myfootnote{\href{https://arxiv.org/abs/1406.2661}{Goodfellow I. J. et al. Generative Adversarial Networks, 2014} \\
		\href{https://arxiv.org/abs/1701.04862}{Arjovsky M., Bottou L. Towards Principled Methods for Training Generative Adversarial Networks, 2017}}
\end{frame}
%=======
\begin{frame}{Recap of previous lecture}
	\begin{figure}
		\centering
		\includegraphics[width=0.8\linewidth]{figs/discrete_wasserstein}
	\end{figure}
	\vspace{-0.3cm}
	\begin{block}{Wasserstein distance}
		\vspace{-0.7cm}
		\[
		W(\pi, p) = \inf_{\gamma \in \Gamma(\pi, p)} \bbE_{(\bx, \by) \sim \gamma} \| \bx - \by \| =  \inf_{\gamma \in \Gamma(\pi, p)} \int \| \bx - \by \| \gamma (\bx, \by) d \bx d \by
		\]
		\vspace{-0.5cm}
		\begin{itemize}
			\item $\gamma(\bx, \by)$ -- transportation plan (the amount of "dirt" that should be transported from point $\bx$ to point $\by$).
			\item $\Gamma(\pi, p)$ -- the set of all joint distributions $\gamma (\bx, \by)$ with marginals $\pi$ and $p$ ($\int \gamma(\bx, \by) d \bx = p(\by)$, $\int \gamma(\bx, \by) d \by = \pi(\bx)$).
			\item $\gamma(\bx, \by)$ -- the amount, $\|\bx - \by \|$-- the distance.
		\end{itemize}
	\end{block}
	\myfootnotewithlink{https://udlbook.github.io/udlbook/}{Simon J.D. Prince. Understanding Deep Learning, 2023}
\end{frame}
%=======
\begin{frame}{Outline}
	\tableofcontents
\end{frame}
%=======
\section{Wasserstein GAN}
%=======
\begin{frame}{Wasserstein GAN}
	\begin{block}{Wasserstein distance}
		\vspace{-0.5cm}
		\[
		W(\pi || p) = \inf_{\gamma \in \Gamma(\pi, p)} \bbE_{(\bx, \by) \sim \gamma} \| \bx - \by \| =  \inf_{\gamma \in \Gamma(\pi, p)} \int \| \bx - \by \| \gamma (\bx, \by) d \bx d \by
		\]
		\vspace{-0.3cm}
	\end{block}
	The infimum across all possible joint distributions in $\Gamma(\pi, p)$ is intractable.
	\begin{block}{Theorem (Kantorovich-Rubinstein duality)}
		\vspace{-0.3cm}
		\[
		W(\pi || p) = \frac{1}{K} \max_{\| f \|_L \leq K} \left[ \bbE_{\pi(\bx)} f(\bx)  - \bbE_{p(\bx)} f(\bx)\right],
		\]
		where $f : \bbR^m \rightarrow \bbR$, $\| f \|_L \leq K$ are $K-$Lipschitz continuous functions ($f: \cX \rightarrow \bbR$)
		\vspace{-0.2cm}
		\[
		|f(\bx_1) - f(\bx_2)| \leq K \| \bx_1 - \bx_2 \|, \quad \text{for all } \bx_1, \bx_2 \in \cX.
		\]
		\vspace{-0.6cm}
	\end{block}
	Now we need only samples to get Monte Carlo estimate for $W(\pi || p)$.
	
	\myfootnotewithlink{https://arxiv.org/abs/1701.07875}{Arjovsky M., Chintala S., Bottou L. Wasserstein GAN, 2017}
\end{frame}
%=======
\begin{frame}{Wasserstein GAN}
	\begin{block}{Theorem (Kantorovich-Rubinstein duality)}
		\[
		W(\pi || p) = \frac{1}{K} \max_{\| f \|_L \leq K} \left[ \bbE_{\pi(\bx)} f(\bx)  - \bbE_{p(\bx)} f(\bx)\right],
		\]
	\end{block}
	\begin{itemize}
		\item Now we have to ensure that $f$ is $K$-Lipschitz continuous.
		\item Let $f_{\bphi}(\bx)$ be a feedforward neural network parametrized by~$\bphi$.
		\item If parameters $\bphi$ lie in a compact set $\boldsymbol{\Phi}$ then $f_{\bphi}(\bx)$ will be $K$-Lipschitz continuous function. 
		\item Let the parameters be clamped to a fixed box $\boldsymbol{\Phi} \in [-c, c]^d$ (e.x. $c = 0.01$) after each gradient update.
	\end{itemize}
	\begin{multline*}
		K \cdot W(\pi || p) = \max_{\| f \|_L \leq K} \left[ \bbE_{\pi(\bx)} f(\bx)  - \bbE_{p(\bx)} f(\bx)\right] \geq \\  \geq \max_{\bphi \in \boldsymbol{\Phi}} \left[ \bbE_{\pi(\bx)} f_{\bphi}(\bx)  - \bbE_{p(\bx)} f_{\bphi}(\bx)\right]
	\end{multline*}
	
	\myfootnotewithlink{https://arxiv.org/abs/1701.07875}{Arjovsky M., Chintala S., Bottou L. Wasserstein GAN, 2017}
\end{frame}
%=======
\begin{frame}{Wasserstein GAN}
	\begin{block}{Standard GAN objective}
		\vspace{-0.2cm}
		\[
		\min_{\btheta} \max_{\bphi} \bbE_{\pi(\bx)} \log D_{\bphi}(\bx) + \bbE_{p(\bz)} \log (1 - D_{\bphi}(\bG_{\btheta}(\bz)))
		\]
		\vspace{-0.2cm}
	\end{block}
	\begin{block}{WGAN objective}
		\vspace{-0.3cm}
		\[
		\min_{\btheta} W(\pi || p) \approx \min_{\btheta} \max_{\bphi \in \boldsymbol{\Phi}} \left[ \bbE_{\pi(\bx)} f_{\bphi}(\bx)  - \bbE_{p(\bz)} f_{\bphi}(\bG_{\btheta}(\bz))\right].
		\]
		\vspace{-0.3cm}
	\end{block}
	\begin{itemize}
		\item Discriminator $D$ is similar to the function $f$, but not the same (it is not a classifier anymore). In the WGAN model, function~$f$ is usually called $\textbf{critic}$.
		\item \textit{"Weight clipping is a clearly terrible way to enforce a Lipschitz constraint"}. If the clipping parameter $c$ is too large, it is hard to train the critic till optimality. If the clipping parameter $c$ is too small, it could lead to vanishing gradients.
	\end{itemize}
	
	\myfootnotewithlink{https://arxiv.org/abs/1701.07875}{Arjovsky M., Chintala S., Bottou L. Wasserstein GAN, 2017}
\end{frame}
%=======
\begin{frame}{Wasserstein GAN}
	\begin{minipage}[t]{0.6\columnwidth}
		\begin{itemize}
			\item WGAN has non-zero gradients for disjoint supports.
			\item $JSD(\pi || p)$ correlates poorly with the sample quality. Stays constast nearly maximum value $\log 2 \approx 0.69$.
			\item $W(\pi || p)$ is highly correlated with the sample quality. 
		\end{itemize}
	\end{minipage}%
	\begin{minipage}[t]{0.4\columnwidth}
		\begin{figure}
			\centering
			\includegraphics[width=\linewidth]{figs/wgan_toy}
		\end{figure}
	\end{minipage}
	\begin{minipage}[t]{0.5\columnwidth}
		\begin{figure}
			\centering
			\includegraphics[width=1.0\linewidth]{figs/dcgan_quality}
		\end{figure}
	\end{minipage}%
	\begin{minipage}[t]{0.5\columnwidth}
		\begin{figure}
			\centering
			\includegraphics[width=1.0\linewidth]{figs/wgan_quality}
		\end{figure}
	\end{minipage}
	
	\myfootnotewithlink{https://arxiv.org/abs/1701.07875}{Arjovsky M., Chintala S., Bottou L. Wasserstein GAN, 2017}
\end{frame}
%=======
\section{f-divergence minimization}
%=======
\begin{frame}{Divergences}
	\begin{itemize}
		\item Forward KL divergence in maximum likelihood estimation.
		\item Reverse KL in variational inference.
		\item JS divergence in standard GAN.
		\item Wasserstein distance in WGAN.
	\end{itemize}
	\begin{block}{What is a divergence?}
		Let $\cP$ be the set of all possible probability distributions. Then $D: \cP \times \cP \rightarrow \bbR$ is a divergence if 
		\begin{itemize}
			\item $D(\pi || p) \geq 0$ for all $\pi, p \in \cP$;
			\item $D(\pi || p) = 0$ if and only if $\pi \equiv p$.
		\end{itemize}
	\end{block}
	\begin{block}{General divergence minimization task}
		\vspace{-0.3cm}
		\[
			\min_p D(\pi || p)
		\]
		\vspace{-0.7cm}
	\end{block}
	\begin{block}{Chalenge}
		We do not know the real distribution $\pi(\bx)$!
	\end{block}
\end{frame}
%=======
\begin{frame}{f-divergence family}
	
	\begin{block}{f-divergence}
		\vspace{-0.3cm}
		\[
		D_f(\pi || p) = \bbE_{p(\bx)}  f\left( \frac{\pi(\bx)}{p(\bx)} \right)  = \int p(\bx) f\left( \frac{\pi(\bx)}{p(\bx)} \right) d \bx.
		\]
		Here $f: \bbR_+ \rightarrow \bbR$ is a convex, lower semicontinuous function satisfying $f(1) = 0$.
	\end{block}
	\begin{figure}
		\centering
		\includegraphics[width=\linewidth]{figs/f_divs}
	\end{figure}
	\myfootnotewithlink{https://arxiv.org/abs/1606.00709}{Nowozin S., Cseke B., Tomioka R. f-GAN: Training Generative Neural Samplers using Variational Divergence Minimization, 2016}
\end{frame}
%=======
\begin{frame}{f-divergence family}
	\vspace{-0.2cm}
	\begin{block}{Fenchel conjugate}
		\vspace{-0.7cm}
		\[
		f^*(t) = \sup_{u \in \text{dom}_f} \left( ut - f(u) \right), \quad f(u) = \sup_{t \in \text{dom}_{f^*}} \left( ut - f^*(t) \right)
		\]
		\vspace{-0.5cm}
	\end{block}
	\textbf{Important property:} $ f^{**} = f$ for convex $f$.
	\begin{block}{f-divergence}
		\vspace{-0.8cm}
		\begin{multline*}
			D_f(\pi || p) = \bbE_{p(\bx)}  f\left( \frac{\pi(\bx)}{p(\bx)} \right)  = \int p(\bx) {\color{violet}f\left( \frac{\pi(\bx)}{p(\bx)} \right) } d \bx = \\ = \int p(\bx) {\color{violet} \sup_{t \in \text{dom}_{f^*}} \left( \frac{\pi(\bx)}{p(\bx)} t - f^*(t) \right)} d \bx = \\ 
			= \int \sup_{t \in \text{dom}_{f^*}} \left( \pi(\bx)t - p(\bx) f^*(t) \right) d \bx .
		\end{multline*}
		\vspace{-0.6cm}
	\end{block}
	Here we seek value of $t$, which gives us maximum value of $ \pi(\bx)t - p(\bx) f^*(t)$, for each data point $\bx$.
	\myfootnotewithlink{https://arxiv.org/abs/1606.00709}{Nowozin S., Cseke B., Tomioka R. f-GAN: Training Generative Neural Samplers using Variational Divergence Minimization, 2016}
\end{frame}
%=======
\begin{frame}{f-divergence family}
	\vspace{-0.4cm}
	\begin{block}{f-divergence}
		\vspace{-0.3cm}
		\[
		D_f(\pi || p) = \bbE_{p(\bx)}  f\left( \frac{\pi(\bx)}{p(\bx)} \right)  = \int p(\bx) f\left( \frac{\pi(\bx)}{p(\bx)} \right) d \bx.
		\]
		\vspace{-0.4cm}
	\end{block}
	\begin{block}{Variational f-divergence estimation}
		\vspace{-0.8cm}
		\begin{multline*}
			D_f(\pi || p)  = {\color{violet}\int} {\color{teal} \sup_{t \in \text{dom}_{f^*}}} \left( \pi(\bx)t - p(\bx) f^*(t) \right) d \bx \geq \\
			 \geq {\color{teal}\sup_{T \in \cT}} {\color{violet}\int} \left( \pi(\bx)T(\bx) - p(\bx) f^*(T(\bx)) \right) d \bx = \\
			 = \sup_{T \in \cT} \left[\bbE_{\pi}T(\bx) -  \bbE_p f^*(T(\bx)) \right]
		\end{multline*}
	\vspace{-0.6cm}
	\end{block}
	This is a lower bound because of Jensen inequality and restricted class of functions $\cT: \cX \rightarrow \bbR$. 
	
	\myfootnotewithlink{https://arxiv.org/abs/1606.00709}{Nowozin S., Cseke B., Tomioka R. f-GAN: Training Generative Neural Samplers using Variational Divergence Minimization, 2016}
\end{frame}
%=======
\begin{frame}{f-divergence family}
	\begin{block}{Variational divergence estimation}
		\[
			D_f(\pi || p) \geq \sup_{T \in \cT} \left[\bbE_{\pi}T(\bx) -  \bbE_p f^*(T(\bx)) \right]
		\]
		The lower bound is tight for $T^*(\bx) = f'\left( \frac{\pi(\bx)}{p(\bx)} \right)$.
	\end{block}
	\begin{block}{Example (JSD)}
		\begin{itemize}
			\item Let define function $f$ and its conjugate $f^*$
			\[ 
				f(u) = u \log u - (u + 1) \log (u + 1), \quad f^*(t) = - \log (1 - e^t).
			\]
			\item Let reparametrize $T(\bx) = \log D(\bx)$.
		\end{itemize}
		\vspace{-0.4cm}
	\end{block}
	\[
		\min_{G} \max_D \left[ \bbE_{\pi(\bx)} \log D(\bx) + \bbE_{p(\bz)} \log (1 - D(\bG(\bz))) \right]
	\]

	\myfootnotewithlink{https://arxiv.org/abs/1606.00709}{Nowozin S., Cseke B., Tomioka R. f-GAN: Training Generative Neural Samplers using Variational Divergence Minimization, 2016}
\end{frame}
%=======
\begin{frame}{f-divergence family}
	\begin{block}{Variational divergence estimation}
		\[
			D_f(\pi || p) \geq \sup_{T \in \cT} \left[\bbE_{\pi}T(\bx) -  \bbE_p f^*(T(\bx)) \right]
		\]
	\end{block}
	\textbf{Note:} To evaluate lower bound we only need samples from $\pi(\bx)$ and $p(\bx)$. Hence, we could fit implicit generative model.
	\begin{figure}
		\centering
		\includegraphics[width=1.0\linewidth]{figs/f_div_results}
	\end{figure}

	\myfootnotewithlink{https://arxiv.org/abs/1606.00709}{Nowozin S., Cseke B., Tomioka R. f-GAN: Training Generative Neural Samplers using Variational Divergence Minimization, 2016}
\end{frame}
%=======
\section{Evaluation of likelihood-free models}
%=======
\begin{frame}{Evaluation of likelihood-free models}
	How to evaluate generative models?
	\begin{block}{Likelihood-based models}
		\begin{itemize}
			\item Split data to train/val/test.
			\item Fit model on the train part.
			\item Tune hyperparameters on the validation part.
			\item Evaluate generalization by reporting likelihoods on the test set.
		\end{itemize}
	\end{block}
	\begin{block}{Not all models have tractable likelihoods}
		\begin{itemize}
			\item VAE: compare ELBO values.
			\item GAN: ???
		\end{itemize}
	\end{block}
\end{frame}
%=======
\begin{frame}{Evaluation of likelihood-free models}
	Let take some pretrained image classification model to get the conditional label distribution $p(y | \bx)$ (e.g. ImageNet classifier).
	\begin{block}{What do we want from samples?}
		\begin{itemize}
			\item \textbf{Sharpness}
			\begin{figure}
				\centering
				\includegraphics[width=0.9\linewidth]{figs/sharpness}
			\end{figure}
			The conditional distribution $p(y | \bx)$ should have low entropy (each image $\bx$ should have distinctly recognizable object).
			\item \textbf{Diversity}
			\begin{figure}
				\centering
				\includegraphics[width=0.9\linewidth]{figs/diversity}
			\end{figure}
			The marginal distribution $p(y) = \int p(y | \bx) p(\bx) d \bx$ should have high entropy (there should be as many classes generated as possible).
		\end{itemize}
	\end{block}
	\myfootnotewithlink{https://deepgenerativemodels.github.io}{image credit: https://deepgenerativemodels.github.io}
\end{frame}
%=======
\subsection{Frechet Inception Distance (FID)}
%=======
\begin{frame}{Frechet Inception Distance (FID)}
	\begin{block}{Wasserstein metric}
		\vspace{-0.2cm}
		\[
		W_s(\pi, p) = \inf_{\gamma \in \Gamma(\pi, p)} \left(\bbE_{(\bx, \by) \sim \gamma} \| \bx - \by \|^s\right)^{1/s}
		\]
		\vspace{-0.6cm}
	\end{block}
	\begin{block}{Theorem}
		If $\pi(\bx) = \cN(\bmu_\pi, \bSigma_\pi)$, $p(\by) = \cN(\bmu_p, \bSigma_p)$, then
		\vspace{-0.3cm}
		\[
		W_2^2(\pi, p) = \| \bmu_{\pi} - \bmu_{p}\|_2^2 + \text{tr} \left[ \bSigma_{\pi} + \bSigma_p - 2 \left(\bSigma_{\pi}^{1/2} \bSigma_p \bSigma_{\pi}^{1/2} \right)^{1/2} \right]
		\]
		\vspace{-0.7cm}
	\end{block}
	\begin{block}{Frechet Inception Distance}
		\vspace{-0.3cm}
		\[
			\text{FID} (\pi, p) =  W_2^2(\pi, p)
		\]
		\vspace{-0.6cm}
	\end{block}
	\begin{itemize}
		\item Representations are the outputs of the intermediate layer from the pretrained classification model.
		\item $\bmu_{\pi}$, $\bSigma_{\pi}$ and $\bmu_{p}$, $\bSigma_p$ are the statistics of the feature representations for the samples from $\pi(\bx)$ and $p(\bx | \btheta)$.
	\end{itemize}
	\myfootnotewithlink{https://arxiv.org/abs/1706.08500}{Heusel M. et al. GANs Trained by a Two Time-Scale Update Rule Converge to a Local Nash Equilibrium, 2017}
\end{frame}
%=======
\begin{frame}{Frechet Inception Distance (FID)}
	\vspace{-0.4cm}
	\[
	\text{FID} (\pi, p) =  \| \bmu_{\pi} - \bmu_{p}\|_2^2 + \text{tr} \left[ \bSigma_{\pi} + \bSigma_p - 2 \left(\bSigma_{\pi}^{1/2} \bSigma_p \bSigma_{\pi}^{1/2} \right)^{1/2} \right]
	\]
	\vspace{-0.3cm}
	\begin{itemize}
		\item Needs a large sample size for evaluation.
		\item Calculation of FID is slow.
		\item High dependence on the pretrained classification model.
		\item Uses the normality assumption!
	\end{itemize}
	\begin{figure}
		\includegraphics[width=0.95\linewidth]{figs/fid_normal}
	\end{figure}
	
	\myfootnotewithlink{https://arxiv.org/abs/2401.09603}{Jayasumana S. et al. Rethinking FID: Towards a Better Evaluation Metric for Image Generation, 2024}
\end{frame}
%=======
\subsection{Maximum Mean Discrepancy (MMD)}
%=======
\begin{frame}{Maximum Mean Discrepancy (MMD)}
	\begin{block}{Theorem}
		$\pi(\bx) = p(\by)$ if and only if $\bbE_{\pi(\bx)} f(\bx) = \bbE_{p(\by)} f(\by)$ for any bounded and continuous $f$.
	\end{block}
	\vspace{-0.3cm}
	\[
		\text{MMD}(\pi, p) = \sup_f \left[ \bbE_{\pi(\bx)} f(\bx) - \bbE_{p(\bx)} f(\bx) \right]
	\]	
	\vspace{-0.3cm}
	\begin{block}{Theorem (Reproducing Kernel Hilbert Space)}
		\vspace{-0.6cm}
		\[
			\text{MMD}^2(\pi, p) = \bbE_{\substack{\bx \sim\pi(\bx) \\ \bx' \sim\pi(\bx)}} k(\bx, \bx') + \bbE_{\substack{\by \sim p(\by) \\ \by' \sim p(\by)}} k(\by, \by') - 2 \bbE_{\substack{\bx \sim\pi(\bx) \\ \by \sim p(\by)}} k(\bx, \by)
		\]
		\begin{itemize}
			\item $k(\bx, \by)$ is a positive definite, symmetric kernel function (for example $k(\bx, \by) = \frac{\exp(\|\bx - \by\|^2)}{\sigma^2}$);
			\item $f(\bx) = \langle f, \phi(\bx) \rangle_{\mathcal{H}}$;
			\item $\phi(\bx) = k(\cdot, \bx)$.
		\end{itemize} 
		\vspace{-0.3cm}
	\end{block}
	\myfootnotewithlink{https://arxiv.org/abs/2401.09603}{Jayasumana S. et al. Rethinking FID: Towards a Better Evaluation Metric for Image Generation, 2024}
\end{frame}
%=======
\begin{frame}{Maximum Mean Discrepancy (MMD)}
		\vspace{-0.6cm}
		\[
			\text{MMD}^2(\pi, p) = \bbE_{\substack{\bx \sim\pi(\bx) \\ \bx' \sim\pi(\bx)}} k(\bx, \bx') + \bbE_{\substack{\by \sim p(\by) \\ \by' \sim p(\by)}} k(\by, \by') - 2 \bbE_{\substack{\bx \sim\pi(\bx) \\ \by \sim p(\by)}} k(\bx, \by)
		\]
		\vspace{-0.3cm}
	\begin{figure}
		\includegraphics[width=0.95\linewidth]{figs/mmd_normal}
	\end{figure}
	\begin{itemize}
		\item Needs less sample size for evaluation.
		\item High dependence on the pretrained classification model.
		\item Works with any distribution!
	\end{itemize}
	\myfootnotewithlink{https://arxiv.org/abs/2401.09603}{Jayasumana S. et al. Rethinking FID: Towards a Better Evaluation Metric for Image Generation, 2024}
\end{frame}
%=======
\subsection{Precision-Recall}
%=======
\begin{frame}{Precision-Recall}
	\begin{block}{What do we want from samples}
		\begin{itemize}
			\item \textbf{Sharpness}: generated samples should be of high quality.
			\item \textbf{Diversity}: their variation should match that observed in the training set.
		\end{itemize}
	\end{block}
	\vspace{-0.5cm}
	\begin{figure}
		\includegraphics[width=0.95\linewidth]{figs/pr_curve}
	\end{figure}
	\vspace{-0.3cm}
	\begin{itemize}
		\item \textbf{Precision} denotes the fraction of generated images that are realistic.
		\item \textbf{Recall} measures the fraction of the training data manifold covered by the generator.
	\end{itemize}
	\myfootnotewithlink{https://arxiv.org/abs/1904.06991}{Kynkäänniemi T. et al. Improved precision and recall metric for assessing generative models, 2019}
\end{frame}
%=======
\begin{frame}{Precision-Recall}
	\begin{itemize}
		\item $\cS_{\pi} = \{\bx_i\}_{i=1}^{n} \sim \pi(\bx)$ -- real samples;
		\item $\cS_{p} = \{\bx_i\}_{i=1}^{n} \sim p(\bx | \btheta)$ -- generated samples.
	\end{itemize}
	Embed samples using pretrained classifier network (as previously):
	\[
		\cG_{\pi} = \{\mathbf{g}_i\}_{i=1}^n, \quad \cG_{p} = \{\mathbf{g}_i\}_{i=1}^n.
	\]
	Define binary function:
	\[
		f(\mathbf{g}, \cG) = 
		\begin{cases}
			1, \quad \text{if exists } \mathbf{g}' \in \cG: \| \mathbf{g}  - \mathbf{g}'\|_2 \leq \| \mathbf{g}' - \text{NN}_k(\mathbf{g}', \cG)\|_2; \\
			0, \quad \text{otherwise.}
		\end{cases}
	\]
	\[
		\text{Precision} (\cG_{\pi}, \cG_{p}) = \frac{1}{n} \sum_{\mathbf{g} \in \cG_{p}} f(\mathbf{g}, \cG_{\pi}); \,\, \text{Recall} (\cG_{\pi}, \cG_{p}) = \frac{1}{n} \sum_{\mathbf{g} \in \cG_{\pi}} f(\mathbf{g}, \cG_{p}).
	\]
	\vspace{-0.4cm}
	\begin{figure}
		\includegraphics[width=0.7\linewidth]{figs/pr_k_nearest}
	\end{figure}
	\myfootnotewithlink{https://arxiv.org/abs/1904.06991}{Kynkäänniemi T. et al. Improved precision and recall metric for assessing generative models, 2019}
\end{frame}
%=======
\begin{frame}{Precision-Recall}
	\begin{figure}
		\includegraphics[width=\linewidth]{figs/pr_vs_fid}
	\end{figure}
	\begin{figure}
		\includegraphics[width=0.75\linewidth]{figs/pr_truncation}
	\end{figure}
	\myfootnotewithlink{https://arxiv.org/abs/1904.06991}{Kynkäänniemi T. et al. Improved precision and recall metric for assessing generative models, 2019}
\end{frame}
%=======
\begin{frame}{Truncation trick}
	\begin{block}{BigGAN: truncated normal sampling}
		\vspace{-0.3cm}
		\[
			p(\bz | \psi) = \cN(\bz | 0, \bI) / \int_{-\infty}^\psi \cN(\bz' | 0, \bI) d\bz'
		\]
		Components of $\bz \sim \cN(0, \bI)$ which fall outside a predefined range are resampled.
	\end{block}
	
	\begin{block}{StyleGAN}
		\vspace{-0.2cm}
		\[
			\bz' = \hat{\bz} + \psi \cdot (\bz - \hat{\bz}), \quad \hat{\bz} = \bbE_{\bz} \bz
		\]
		\vspace{-0.2cm}
		\begin{itemize}
			\item Constant $\psi$ is a tradeoff between diversity and fidelity. 
			\item $\psi=0.7$ is used for most of the results.
		\end{itemize}
	\end{block}

	\myfootnote{\href{https://arxiv.org/abs/1809.11096}{Brock A., Donahue J., Simonyan K. Large Scale GAN Training for High Fidelity Natural Image Synthesis, 2018} \\
	\href{https://arxiv.org/abs/1812.04948}{Karras T., Laine S., Aila T. A Style-Based Generator Architecture for Generative Adversarial Networks, 2018}}
\end{frame}
%=======
\begin{frame}{Summary}
	\begin{itemize}
		\item Kantorovich-Rubinstein duality gives the way to calculate the EM distance using only samples.
		\vfill
		\item Wasserstein GAN uses Kantorovich-Rubinstein duality for getting Earth Mover distance as model objective.
		\vfill
		\item f-divergence family is a unified framework for divergence minimization, which uses variational approximation. Standard GAN is a special case of it.
		\vfill
		\item We need a measure of quality for the implicit models (like GANs). 
		\vfill 
		\item Frechet Inception Distance is the most popular metric for the implicit models evaluation. 
		\vfill
		\item Maximum Mean Discrepancy tries to fix some of the FID drawbacks.
		\vfill
		\item Precision-recall allow to select model that compromises the sample quality and the sample diversity.	
		\vfill 
		\item Truncation tricks help to select model with the compromised samples: diverse and sharp.
	\end{itemize}
\end{frame}
\end{document} 